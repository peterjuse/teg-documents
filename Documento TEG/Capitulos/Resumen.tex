%-----------------------------------------------------------------------------
%	Resumen
%-----------------------------------------------------------------------------
\chapter*{Resumen}
\addcontentsline{toc}{chapter}{Resumen}
\markboth{Resumen}{Resumen} 
En general la adopción de nuevas tecnologías suele ocurrir de manera dispar. En algunas ocasiones la adopción es lenta y paulatina lo cual permite que pueda madurar en los diversos entornos en donde se implantan así como también permite crear formas ordenadas y planificadas de crecimiento de los elementos que se encuentran involucrados. Por otro lado también existen tecnologías que debido a su rápido crecimiento hacen que las personas y organizaciones deban adaptarse y ser flexibles en la manera en que se piensa se deben usar los avances, así como también la gama completa de oportunidades y debilidades que representa su uso.\\ 

Una de esas tecnologías que ha cambiado la manera en que los seres humanos actuamos es el computador personal. Con el acceso a una plataforma tan poderosa como lo es el computador, la capacidad de poder automatizar elementos de la vida cotidiana y de procesos complejos en las industrias, se tiene la receta para ser una de las herramientas mas importantes que haya creado el hombre. Es difícil siquiera pensar en la actualidad como para incontables tareas dependemos ahora del computador. Una arista importante de dichos elementos es brindar la capacidad de computo a otros artefactos que antes no lo poseían. La arquitectura del computador moderno es lo suficientemente flexible para encontrarnos con cada vez más dispositivos que son capaces de realizar tareas aun sin la necesidad de intervención alguna.\\

Otra tecnología que ha cambiado al mundo es la capacidad de acceder y compartir información a través de una red. Su evolución a lo largo del tiempo a lo que ahora es el internet ha sido uno de los avances cruciales en la historia. No es una tecnología reciente, pero se ha masificado y democratizado su acceso de tal forma que es un aspecto omnipresente para mas de la mitad de la población mundial. Las diversas plataformas que se apoyan en la "red de redes" nos han ayudado a masificar la adopción de otro conjunto enorme de otras tecnologías, pues su flexibilidad y la madurez de los procesos que involucran la capacidad de conexión es catalizador de oportunidades para simplificar muchos aspectos de la vida hasta ahora no disponible.\\

Si juntamos los aspectos de computo y conectividad vemos que de manera disruptiva actualmente se tiene la oportunidad de mejorar y automatizar muchos de los procesos que antes por costo, logística o complejidad eran difíciles de llevar a cabo. Casi todos ahora tenemos la posibilidad de conectarnos a internet con un computador que cabe en nuestros bolsillos y poder realizar todo tipo de tareas complejas con ello. El crecimiento en la información y en masificación de artefactos y elementos que obtienen datos de su entorno proveen a los involucrados de una nueva visión del funcionamiento de las cosas es evidente cuando cada vez dichos artefactos adquieren esas capacidades de computo y conexión que antes eran impensables. A esta revolución de la información la llamamos "internet de las cosas" y es una tendencia en la tecnología en pleno crecimiento.\\ 

Sin embargo, este nuevo concepto en la manera de usar estas tecnologías no ha venido sin presentar retos y dificultades propias de cada avance. El gran volumen de información generado de manera automatizada, el control y monitoreo de dispositivos y artefactos a lo largo y ancho de complejos sistemas y y nuevos flujos automáticos donde antes no eran posibles de realizar hacen cada vez mas difícil el poder tener un panorama claro de las operaciones de estos sistemas por lo que se requieren de infraestructuras, plataformas y desarrollos nuevos para poder mejorar los aspectos de adopción mas ordenada de una forma tan nueva de hacer las cosas. Es así como nace la propuesta de comenzar a realizar la integración de tecnologías probadas que juntas puedan dar un mejor panorama en la observación y control de elementos de las operaciones y acciones que llevamos a cabo de manera automatizada en nuestro día a día.\\

El potencial que representa el internet de las cosas estará incompleto sin sistemas que sean capaces de brindar de manera adecuada información útil a los usuarios y la posibilidad de brindar la capacidad de controlar flujos automáticos de manera simple, se que este desencadene una serie de actividades en nuestro hogar o sea un complejo conjunto de operaciones criticas en una industria. A continuación se presenta una propuesta para comenzar a llenar ese vacío, mostrando el desarrollo e investigación realizado para crear una herramienta de visualización de datos de sensores de dispositivos, así como también la posibilidad de poder controlar los elementos configurables (actuadores) a través de la creación de flujos de automatización. Por tales objetivos a está software desarrollado lleva el nombre de "HAMACA" que no es más que el acronimo de "Herramienta de Automatización, Monitoreo y Análisis de Componentes y Artefactos", como tanto nos gusta a los desarrolladores usar para nombrar nuestros proyectos.\\
 
\vspace{\fill}
Palabras Claves: Internet, Computador, Internet de las Cosas, Sistemas automatizados, Monitoreo, Visualización.
\vspace{20px}