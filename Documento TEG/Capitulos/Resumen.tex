%-----------------------------------------------------------------------------
%	Resumen
%-----------------------------------------------------------------------------
\chapter*{Resumen}
\addcontentsline{toc}{chapter}{Resumen}
\markboth{Resumen}{Resumen}
La ubicuidad de una creciente cantidad de dispositivos inteligentes que se pueden conectar e interactuar con otros sistemas y dispositivos, aunado a la presencia de cada vez más sensores capaces de leer múltiples variables como de actuadores que se accionen siguiendo un comportamiento preestablecido han generado la necesidad de poder tratar con la complejidad subyacente tanto del volumen de datos que se generan en el proceso, así como también de formas adecuadas de presentar la información capturada de dichos procesos, sin dejar de lado la gestión de dichos procesos automatizados de una manera sencilla.\\

A esta situación donde se unen las capacidades de computo, de captura de información y de actuar sobre el ambiente, en conjunto con la posibilidad de conectarse a las redes es lo que ahora llamamos al internet de las cosas (o IoT por sus siglas en ingles). El potencial que representa el internet de las cosas en la actualidad aunque amplio puede verse incompleto sin sistemas que sean capaces de brindar de manera adecuada información útil a los usuarios y la posibilidad de brindar la capacidad de controlar flujos automáticos de manera simple y centralizada ya sea que este desencadene una serie de actividades en nuestro hogar o sea un complejo conjunto de operaciones criticas en una industria.\\ 

A continuación se presenta una propuesta para comenzar a llenar ese vacío fundamental entre la observabilidad y la gestión de dichos dispositivos IoT, mostrando el desarrollo e investigación realizado para crear una herramienta de visualización de datos de sensores, así como también la posibilidad de poder controlar los elementos configurables (actuadores) a través de la creación de flujos de automatización. Por tales objetivos a está software desarrollado lleva el nombre de ``HAMACA" que no es más que el acronimo de ``Herramienta de Automatización, Monitoreo y Análisis de Componentes y Artefactos", como tanto nos gusta a los desarrolladores usar para nombrar nuestros proyectos.\\
 
\vspace{\fill}
Palabras Claves: Internet, Computador, Internet de las Cosas, Sistemas automatizados, Monitoreo, Visualización.
\vspace{10px}