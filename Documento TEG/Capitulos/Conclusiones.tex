%-----------------------------------------------------------------------------
%	 Conclusiones
%-----------------------------------------------------------------------------

\lhead[\thepage]{Conclusiones \thechapter. \rightmark}
\rhead[Conclusiones \thechapter. \leftmark]{\thepage}

%	Capitulo N: Conclusiones
\chapter{Conclusiones}
\markboth{Conclusiones}{Conclusiones}

Este trabajo de investigación permitió demostrar la factibilidad del desarrollo de una herramienta de visualización de datos, monitoreo, control y automatización de procesos concernientes a dispositivos de Internet de las cosas, partiendo desde prototipos funcionales que pudiesen generar información para llevarlo a un escenario lo más realista posible sobre el como de desplegaría un futuro contexto donde los artefactos inteligentes son ubicuos.\\

El desarrollo de prototipos de dispositivos IoT permitió identificar las fortalezas y debilidades de los enfoques utilizados para medir y controlar estos artefactos de manera transparente. Se observaron como el fenómeno de la variedad entre sensores, actuadores y dispositivos en si pueden generar data que aunque representen lo mismo tiene matices de eficiencia y eficacia que pueden ser aprovechados por otros elementos en la búsqueda de capturar, mostrar y utilizar la información que capturamos de ellos.\\

Por el lado del desarrollo de la aplicación web se demostró que un software que pueda orquestar e integrar herramientas existentes para tareas especificas como visualización de datos, automatización de procesos, monitoreo y control de dispositivos puede ser una alternativa viable para poder centralizar las operaciones inherentes a dispositivos IoT y que es un esquema flexible que permite no solo adaptarse a las necesidades de los usuarios sino también con alta capacidad de crecimiento futuro, sea desde el punto de vista de protocolos y estándares, como de software análisis automatizado.\\

Finalmente, se considera que los objetivos planteados en el marco del planteamiento del problema fueron cumplidos con éxito y se espera que este software sea al final de utilidad en donde el contexto permita aprovechar las funcionalidades desarrolladas anteriormente, sea dentro del ámbito académico, investigativo, industrial o uso personal.

