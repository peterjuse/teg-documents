%-----------------------------------------------------------------------------
%	 Problema de Investigación
%-----------------------------------------------------------------------------

\lhead[\thepage]{Problema de Investigación \thechapter. \rightmark}
\rhead[Problema de Investigación \thechapter. \leftmark]{\thepage}

%	Capitulo 4: Problema de Investigación
\chapter{Problema de Investigación}
\markboth{Problema de Investigación}{Problema de Investigación}

%	Sección uno: Planteamiento del Problema
\section{Planteamiento del Problema}
\lhead[\thepage]{\thesection. Planteamiento del Problema}
Cada día es mas común el poder observar dispositivos y artefactos que consideramos normales, adquirir capacidades de conexión, de captura de datos y capaces de ejercer acciones sobre el ambiente. A estos dispositivos son parte de un boom tecnológico al cual se le llama Internet de las Cosas y su importancia radica en que cambiaran la manera con la que interactuamos con los ambientes en donde hacemos vida, como nuestros, trabajos, hogares y ciudades.\\

El Internet de las Cosas abre la puerta a un grupo importante de innovaciones a todo nivel y un lugar que se beneficiará de esto serán los hogares.  La idea de tener artefactos, dispositivos y electrodomésticos que sean cada vez mas autónomos no es nueva, pero la domótica ya esta teniendo una evolución en cuanto a la manera en que estos no solo adquieren capacidad de comunicarse con otros sistemas sino como interactuamos con nuestros hogares, con el fin de automatizar aquellas labores tediosas y difíciles, de proveer de mayores niveles de seguridad a nuestros bienes y a nuestros seres queridos y finalmente interactuar de maneras ineditas.\\

Es así como los procesos del hogar se van automatizando gracias a estos artefactos inteligentes y que son capaces de medir variables de interés, como el consumo de los recursos del hogar, alertas de sensores de vigilancia o aquellos datos generados por experiencias de usuario personalizadas. En la actualidad existen sistemas y plataformas (tanto libres como propietarias) que permiten conocer el estado actual de las métricas obtenidas de los dispositivos y controlarlos usando recetas de comportamiento preestablecidas y programables.\\ 

Una cualidad importante de estos sistemas es que los dispositivos y artefactos conectados aportan una gran cantidad de datos que pueden ser útiles para la toma de decisiones tanto automatizada como no automatizada. Sin embargo, un reto que debe afrontarse es que ante la creciente cantidad de dispositivos IoT que se suman a los sistemas domóticos se necesitan sistemas que sean capaces no solamente de mostrar una instantánea del estado actual de los sensores, actuadores y dispositivos en general en el hogar, sino el poder almacenar y procesar todos los datos (muchos de ellos semi-estructurados o no estructurados) que se producen de manera continua, con el fin de encontrar información que pueda ser utilizada tanto para mejorar los proceso en los que se busca determinar patrones de consumo, puntos de falla y optimizar el manejo y control de los aspectos automatizables del hogar.\\

Dadas las características y complejidad que implica este problema, surgen las preguntas de ¿cual es la mejor arquitectura para un sistema domótico que aproveche todos los datos que se generan?, ¿requieren estos datos algún tratamiento especial para su almacenamiento?, ¿qué información es posible obtener de los datos recabados de los dispositivos y sus procesos a través de un sistema domótico y que conocimiento nos pueden dejar?, y finalmente ¿es posible que al tener ese conocimiento se pueda realizar una toma de decisiones automatizadas que tengan en cuenta inteligentemente las condiciones actuales del ambiente, así como también los gustos de las personas, de forma de ajustar el funcionamiento de los dispositivos?

\subsection{Justificación}
Los datos generados por el uso y consumo de recursos, por las estadísticas y operaciones de sistemas domóticos en la actualidad tiene un carácter comercial poco explorado. Muchos de los recursos utilizados en los hogares no cuentan con inspección o solo se obtienen métricas de forma global a través de la facturación de los proveedores de los servicios. Esta tendencia está cambiando poco a poco pero abre la oportunidad de ofrecer tanto en hogares construidos o como en los diseños de los nuevos hogares inteligentes, de forma que puedan aportar características de eficiencia energética, de consumo de recursos vitales y su reposición sustentable y de mayores niveles de seguridad y confort para los habitantes.\\

La respuesta a diversas preguntas planteadas en la investigación pasa por la creación una herramienta analítica diseñada para tratar con grandes volúmenes de datos, que sea software libre y capaz de integrarse con soluciones existentes utilizadas para monitorear y controlar dispositivos IoT, con la cual se puedan obtener modelos que revelen patrones de consumo y uso sobre uno o mas procesos automatizados o automatizables en el hogar.

\subsection{Alcance}
Como se explico en la sección automatización de hogar existen cuatro posible campos de acción automatizables del hogar que son la administración y monitoreo de recursos (energía, luz, HVACS, agua, desperdicios, etc), la seguridad del hogar, la accesibilidad y el uso de sistemas para el entretenimiento y confort de las personas. Con esto en mente se plantea la creación de una herramienta de análisis de datos al cual se pueda integrar con  un sistema de automatización de hogares, que controle y registre datos de uno o mas dispositivos cuyos sensores sobre uno los campos de acción de automatización de hogar, que ademas sea fácil de escalar y con un costo bajo de implementación. 

\subsection{Objetivos}
\subsubsection{Objetivos Generales}
El objetivo de esta investigación es el poder determinar la mejor estrategia para afrontar el desarrollo de un sistema de domótico inteligente y adaptado al Internet de las Cosas, con la capacidad de analizar los datos históricos y actuales del hogar para optimizar los procesos automatizados, haciendo uso de técnicas de Big data.
\subsubsection{Objetivos Específicos}
Para alcanzar el objetivo general estipulado se plantean los siguientes objetivos específicos:
\begin{itemize}
\item Utilizar la metodología Scrum durante el diseño y creación de las herramientas durante todo el proceso. 
\item Establecer la arquitectura adecuada para dar soporte a un sistema domótico y el software de análisis de los datos de dispositivos IoT.
\item Seleccionar el software de gestión domótico adecuado y que permita la integración de la herramienta de análisis a crear.
\item Seleccionar y utilizar herramientas diseñadas para trabajar con grandes volúmenes de datos y construir un pequeño cluster en el cual desplegarlas.
\item Construir prototipos funcionales de uno o mas dispositivos IoT para la captura de datos de las variables sobre los procesos que están monitoreando y automatizando.
\item Realizar la captura de datos y la construcción de modelos bajo metodología KDD y Fundamental Methodology for Data Science para el análisis de los mismos y generar conocimiento útil para los usuarios.
\item Establecer comportamientos dinámicos en base al conocimiento obtenido en los prototipos funcionales.
\item Utilizar herramientas de visualización de datos adecuadas para informar los resultados obtenidos y realizar sugerencias.
\end{itemize}
