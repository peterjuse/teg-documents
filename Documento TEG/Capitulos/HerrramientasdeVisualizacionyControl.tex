%-----------------------------------------------------------------------------
%	 Herramientas de Monitoreo, Visualización y Control
%-----------------------------------------------------------------------------

\lhead[\thepage]{Herramientas de Monitoreo, Visualización y Control \thechapter. \rightmark}
\rhead[Herramientas de Monitoreo, Visualización y Control \thechapter. \leftmark]{\thepage}

%	Capitulo 4: Herramientas de Monitoreo, Visualización y Control
\chapter{Herramientas de Monitoreo, Visualización y Control}
\markboth{Herramientas de Visualización y Control}{Herramientas de Monitoreo, Visualización y Control}
Si el internet de las cosas es todo sobre la capacidad de conectar dispositivos con sensores y actuadores a otros dispositivos o internet para lograr uno más fines, el sentido con el que operan estos dispositivos es gracias a la generación de datos, información y de conocimiento de manera transversal a las partes implicadas. ¿Pero como podemos realmente aprovechar a los dispositivos y las ventajas que nos traen si no somos capaces de entender que datos nos proveen?.\\

A principios de este milenio, en el mundo de la tecnología se empezó a gestar uno de los paradigmas mas importantes en el área: Con los datos cada vez mas faciles de obtener, quienes fuesen capaces de interpretarlos de mejor manera y convertirlos en información útil serían quienes llevarían la punta de lanza de los nuevos desarrollos por venir. En el año 2006, Clive Humby, considerado como uno de los primeros científicos de datos, acuño la frase: ``los datos son el nuevo petroleo". Luego Micheal Palmer completo esta idea diciendo ``Los datos son  valiosos, pero si no están refinados, en realidad no se pueden usar\cite{datosPetroleo}.\\

Si examinamos detenidamente el flujo de cualquier proceso de creación de conocimiento, esta afirmación se ve respaldada plenamente. Deben existir procesos intermedios en donde estos datos se conviertan en información útil para quienes observan dichos procesos esperando responder o resolver problema, necesidad o inclusive encontrar las oportunidades que otros no han visto, ventajas que en el mundo competitivo de hoy son fundamentales. Una manera de hacer esto es a través de ordenar los datos para mostrarse basados en una lógica. Este paradigma de tratamiento de la información es natural para el ser humano, por lo que saber aprovechar de manera correcta el como presentar datos a contrapartes para respaldar una hipótesis es una forma muy efectiva de generar valor a los datos sobre un particular.\\

Ahora supongamos que los datos que observamos presentan un panorama en tiempo real de características de fenómenos y que estos de momento no persiguen actividades pasadas. El representar esos datos para hacer seguimiento del fenómeno adquiere una dimensión de vigilancia. Lamentablemente los seres humanos no son aptos para ser participe de vigilancia de datos e información a través del tiempo, pero teniendo herramientas de visualización adecuados así como ciertas rutinas establecidas según se demande en el caso caemos en el concepto de monitoreo. Esta actividad cuya faceta es en muchos aspectos pasiva puede llegar a ser tediosa por lo que contar con indicadores que ayuden a ver los cambios de estados de los fenómenos observados es vital en muchos ámbitos.\\

Pero, ¿que sucede si ese monitoreo nos remite a una situación en donde debamos actuar?, ¿como podemos cambiar las características de lo que está ocurriendo cuando muchas veces donde ocurre no es siquiera el mismo lugar geográfico donde se encuentra el fenómeno en sí? Es allí donde surgen herramientas cuyo fin es poder operar uno o múltiples flujos de tareas acordes a recetas existentes o requiriendo la intervención humana para establecer la consecución de un objetivo sobre la situación establecida. A estas herramientas las llamamos herramientas de control\\

Muchas veces encontramos herramientas que cumplen muy bien uno o incluso dos de los aspectos mencionados anteriormente. Por ejemplo, es muy común que las herramientas de visualización permitan en alguna medida el monitoreo de datos en tiempo real, o herramientas de control permitan tener la capacidad de monitorear algunos procesos de manera general.\\

La importancia de reconocer cual herramienta es la correcta para un caso u otro en particular significa que la obtención de conocimiento, la vigilancia y la gestión de procesos se realmente eficiente, aumentando el valor de las tareas que se realizan, independientemente del área al cual se aplique. A continuación se presentan de manera mas detallada las características de estas herramientas y como estos son utilizados en la actualidad en consonancia con el problema de investigación planteado.

\section{Herramientas de Visualización}
Una de las maneras en que se puede inferir un conocimiento nuevo es a través de la información que se puede desplegar de manera visual. Los seres humanos están acostumbrados a que de esta forma podamos entender mejor nuestro ambiente por lo que es natural pensar que a partir de un conjunto de datos que se presenten de manera ordenada y lógica podemos ser capaces de comprender elementos que no necesariamente se pueden ver a simple vista. Así podemos definir como la visualización de datos como la  representación gráfica de información y datos. Al utilizar elementos visuales como cuadros, gráficos y mapas\cite{visualizaciondef}.\\

Es así como el ser humano desde la antigüedad ha buscado las mejores formas de representar esos datos e información. Disciplinas como la matemática han permitido a las personas atravesar la barrera de datos inconexos a  útil y accionable. En la actualidad la gran disponibilidad de los datos hace que cada vez mas necesitemos de herramientas que no permitan resumir de forma adecuada y gráfica la descripción de uno o múltiples fenómenos de los cuales se plantea la extracción de la información. Es como las herramientas de visualización de datos proporcionan una manera accesible de ver y comprender tendencias, valores atípicos y patrones en los datos.\\

Desde el punto de vista computacional las herramientas de visualización son cada vez más importante para darle sentido a las billones de filas de datos que se generan cada día. La visualización de datos ayuda a contar historias seleccionando los datos en una forma más fácil de entender\cite{visualizaciondef} siempre recordando que el objetivo final es poder transmitir el análisis de dichos elementos a un tercero.\\

Para el caso de la investigación se ha de tener en cuenta que buscamos una forma de poder observar datos que ingresan de múltiples fuentes y características, teniendo en cuenta que los datos de los sensores de dispositivos IoT en realidad no son homogéneos y por lo tanto, los usuarios que deseen ver esos datos deben ser capaces de poder comprenderlos haciendo uso de cualquiera de los métodos existentes para visualizarlos. 

\subsection{Métodos utilizados para visualizar datos}
Independientemente de la herramientas de visualización, existen un conjunto enorme de formas de representar los datos, dependiendo de las características que estos posean. Es importante señalar el aspecto de poder seleccionar de manera correcta el método adecuado para representarlos es una elección critica, pues en una etapa de análisis, equivocarse puede representar una conclusión errónea.\\

Entre los diversos métodos de visualización tenemos:
\begin{itemize}
\item Gráficos: Las visualizaciones gráficas informan de comparaciones, relaciones y tendencias. Resaltan y hacen más claras las cifras.\cite{ibmviz} La representación gráfica permite establecer valores que no se han obtenido experimentalmente sino mediante la interpolación (lectura entre puntos) y la extrapolación (valores fuera del intervalo experimental). Entre los mas comunes tipos de gráficos están los gráficos en columnas, gráficos en barras, gráficos de linea, gráficos de área, gráficos circulares (torta o donut), entre otros.  
\item Tablas: Las tablas son formas ordenadas de datos en los cuales se busca categorizarlos basados en su posición por filas, por columnas o por ambas.
\item Mapas: Un mapa no es mas que la representación geográfica simplificada dando información relativa al plano observado.Generalmente a dos dimensiones se suelen representar las diferencias entre características usando colores.
\item Infografías: Las infografías son diagramas visuales complejos que aprovechan recursos como imágenes, texto, diagramas, logos, etc, para poder establecer de manera didáctica una idea de una manera dinámica.\
\item Cuadros de mando: Es un conjunto de indicadores representados de manera agrupada generalmente usados para representar resultados (parciales o totales) sobre datos a los que ya se las ha realizado un análisis previo. Busca representar solo la información indispensable para responder una serie de preguntas especificas.   
\end{itemize}


\section{Herramientas de Monitoreo}
Las herramientas de monitoreo son aquellas que se basan en el proceso continuo y sistemático en el cual se verifican la eficiencia y la eficacia de parámetros de uno o más elemento mediante la identificación de comportamientos o de la presentación de datos para detectar eventuales anomalías.\\

Para el proceso de monitoreo es importante entender que la vigilancia del fenómeno observado es plenamente identificable con los datos recibidos, y muchos veces estos son representados con información gráfica de manera similar a la manera que lo hacen las herramientas de visualización de datos. Sin embargo es importante entender que la visualización de datos aunque se puede dar en tiempo real, esta mas pensando para entender el pasado en base los datos mismos. El monitoreo entiende que existe previamente un conocimiento con el cual se debe regir la observación para identificar las anomalías antes mencionadas.\\

En general las herramientas de monitoreo también sirven para automatizar situaciones criticas o realizar alertas basadas en comportamientos inesperados en la data que esta ingresando. Dichas alertas pueden verse representadas en diversas formas como indicadores visuales en un panel, correos electrónicos, mensajería instantánea o cualquier otro método que este integrado y disponible para llamar la atención de los responsables e incluso permitiendo que otros sistemas puedan actuar de manera automática en respuesta a la eventualidad. 

En el marco de está investigación se requiere que la herramienta sea capaz de poder hacer el seguimiento continuo de múltiples variables sobre sensores y actuadores registrados. Esta información puede ser usada para crear alertas automáticas basadas en los patrones reconocidos anteriormente en el contexto de del dispositivo IoT. 


\section{Herramientas de Control}
Las herramientas de control difieren de las herramientas anteriores porque estos no buscan mostrar datos e información al usuario final, sino que buscan establecer formas de gestión de procesos y modificar basado en una o una serie de reglas el comportamiento de otro artefacto. Estas herramientas permiten planificar tareas y procedimientos de manera previa para un escenario dado, así como también permiten que las personas intervengan en escenarios donde se requiere del conocimiento experto no programado de una automatización.\\

En particular los software de control son todos aquellos sistemas o programas que se encuentras desplegados a lo largo o ancho de las operaciones de un flujo de trabajo y son omnipresentes en la actualidad de cualquier industria o empresa y buscan la menor intervención posible de las personas para evitar introducir el componente del error humano, así como también orquestar las acciones de procesos tediosos o costosos en recursos.\\

Una herramienta de control robusta debe ser capaz no solo llevar los procesos como fueron establecidos y automatizados sino también dar la oportunidad a los usuarios finales de tomar el mando en cualquier momento y en el mejor de los escenarios, ser lo suficientemente flexible para poder generar nuevos flujos o modificar los existentes de manera que todo el proceso sea parametrizable.\\ 

Finalmente, para el caso de los dispositivos IoT una herramienta de control debe soportar estándares y protocolos con los cuales este garantizada la interoperatividad de los sensores y actuadores, de forma que estos sean capaces de recibir ordenes de una forma centralizada acorde a las necesidades del usuario. También el fin de poder controlar los dispositivos es también generar procesos automatizados complejos a partir de programación que no necesariamente existe como regla de negocio dentro del dispositivo sino que se pueda dar a partir de los requerimientos de un usuario final.  
