%-----------------------------------------------------------------------------
%	 Resultados, Limitaciones y Trabajos Futuros
%-----------------------------------------------------------------------------

\lhead[\thepage]{Resultados, Limitaciones y Trabajos Futuros \thechapter. \rightmark}
\rhead[Resultados, Limitaciones y Trabajos Futuros \thechapter. \leftmark]{\thepage}

%	Capitulo 9: Resultados, Limitaciones y Trabajos Futuros
\chapter{Resultados, Limitaciones y Trabajos Futuros}
\markboth{Resultados, Limitaciones y Trabajos Futuros}{Resultados, Limitaciones y Trabajos Futuros}

\section{Resultados}
\lhead[\thepage]{\thesection. Resultados}
Como primer elemento representativo de este trabajo, se hicieron tres prototipos funcionales de dispositivos IoT utilizando diferentes placas programables, entre las que se encuentran el microcontrolador Arduino Uno, Raspberry Pi Zero y Raspberry Pi 3 Modelo B. A cada uno de ellos se le integró una serie de sensores capaces de hacer lecturas sobre el ambiente, así como tambien de elementos actuadores para reaccionar ante determinados eventos tanto de manera automática como de manera manual.\\

Para ello se crearon una serie de scripts en el lenguaje de programación Python y en el lenguaje de programación Arduino, para establecer la comunicación entre sensores y actuadores y las placas programables.  Se estableció también MQTT como protocolo adecuado para realizar la comunicación entre los dispositivos, usando el paradigma de publicación/suscripción ideal para ambientes distribuidos y diseñado para ser liviano y eficiente, métricas que son de relevancia a la hora de trabajar con dispositivos electrónicos con limitaciones de recursos, tanto computacionales como energéticos.\\

La operación de dichos prototipos condujo a la obtención de 2.6 GB de datos no estructurados, 3.1 GB de imágenes y de 1.7 GB de vídeos, que ayudaron a validar la hipótesis de que aun dentro del ambiente de pruebas que se diseñó fue un escenario cercano a la realidad en donde se obtuvo volumen, diversidad de datos así como el modo de operaciones de cada uno de los distintos dispositivos para modificar el ambiente a través de los actuadores.\\

Esto también ayudó a validar la serie de dificultades técnicas que se debían abordar posteriormente para la implementar de manera correcta la solución que permitiese realizar observación, monitoreo y automatización de los dispositivos y afectó a la selección del software para almacenamiento de datos, las integraciones con herramientas de visualización, monitoreo y control disponibles.\\

Llegados al escenario del desarrollo del software que permitiese lograr los objetivos anteriormente expuestos durante el planteamiento del problema se diseñó y desarrollo una solución cuya arquitectura contuviese los elementos necesarios para poder ser desplegado en cualquier plataforma. Esta arquitectura posee el componente de gestión de protocolo de comunicación MQTT bajo la implementación Mosquitto. Gracias al conocimiento adquirido de la naturaleza de los datos se tomó la decisión de usar la base de datos InfluxDB, la cual esta diseñado con el propósito de ser usado bajo el esquema de series de tiempo, dimensión común junto a los valores que proveen los sensores y actuadores a lo largo y ancho del proyecto.\\

Como se mencionó antes, la elección de las herramientas de visualización, monitoreo y control fueron influenciadas también por las características de los datos y de los prototipos. Por el lado de la herramienta de visualización se escogió el uso de Grafana como software de visualización por su sencilles y a la vez potencia para poder desplegar cuadro de mandos a partir de información en tiempo real e histórica de manera intuitiva y que apoye al análisis posterior de los datos que se van generando por parte del usuario final. Por el lado del monitoreo y control, la decisión tomada fue usar Node-Red, un software que permite crear automatizaciones sobre elementos virtuales y físicos de manera visual lo que la hace bastante sencilla de utilizar, ademas de permitir desplegar y realizar acciones en los dispositivos a través de flujos de tareas de forma automática o manual (en los paneles de control  generados a partir de los mismos).\\

La aplicación web fue diseñada para poder integrar estas herramientas así como también para ser modular y fue construida usando el lenguaje de programación Python bajo el framework de desarrollo web Django. Se comenzó el desarrollo desde el módulo de gestión de las configuraciones de la aplicación web, pasando luego por los módulos que integrarían las herramientas de visualización de datos y de monitoreo y control de los dispositivos.\\

Con la data ya disponible se validó que las herramientas fueran capaces de poder hacer uso de los mismos lo cual se evidenció en la creación de en el caso de Grafana, de cuadros de mando con indicadores de las variables medidas por los sensores tanto de forma histórica como los datos en tiempo real y para el caso de Node-red, la creación de flujos de tareas que fuesen capaces de automatizar procesos que no estaban dentro de la lógica realizada para los prototipos de dispositivos IoT como la generación de interfaces de control para encender o apagar elementos actuadores, así como también la posibilidad de usar otro tipos de indicadores en los dashboards que se generan a partir de los flujos de tareas automatizadas (y capaces de generar alertas personalizadas). Por último con respecto a la red se validó que el proyecto fuese capaz de realizar el seguimiento estadístico de las comunicaciones realizadas por dispositivo y por los sensores y actuadores involucrados, llevando y mostrando un registro de las actividades actuales con base a la red.\\

De esta manera se logró realizar un software que es capaz de poder minimizar el impacto de no tener herramientas diseñadas específicamente para el entorno de administración y gestión de dispositivos IoT teniendo en mente el poder ver y actuar de manera centralizada que se encuentren desplegados dentro de un ambiente.

\section{Limitaciones}
\lhead[\thepage]{\thesection. Limitaciones}
A pesar de que se dio una respuesta para poder organizar y gestionar datos y procesos administración de dispositivos IoT, aun existen problemas para hacer de la experiencia de usuario algo más placentero e intuitivo. Las herramientas integradas (Grafana y Node-Red) si bien son bastante fáciles de usar, requieren de cierto conocimiento técnico para poder obtener todas las ventajas que pueden proveer, lo cual puede desalentar al usuario final del uso completo de toda la aplicación.\\

Por otro lado, las fuentes de datos (entiéndase los dispositivos que generan los datos) pueden cumplir con estándares que no están considerados en la actualidad dentro de esta solución por lo que de momento quedarían por fuera de la capacidades actuales del software desarrollado.

\section{Contribuciones}
\lhead[\thepage]{\thesection. Contribuciones}
Durante el presente trabajo de investigación se pudo realizar un desarrollo que de principio a fin, trato de examinar la problemática y sus características en un entorno que fuera lo más cercano a la realidad, teniendo como objetivo en mente que este software pueda ser desplegado con confianza en entornos donde se requiere la observación y gestión de dispositivos IoT.\\

En principio, el presente trabajo se consideró solo para el uso de este software dentro de ámbitos doméstico, es decir, en hogares con alta cantidad de artefactos inteligentes. Sin embargo, después de examinar detenidamente otros escenarios se pudo establecer que este software se puede utilizar en casi cualquier contexto que requiera de una forma de observar datos de dispositivos, así como poder hacer gestión de automatización de procesos, como bien puede ser el contexto de un laboratorio (bien sea de índole académico, industrial o comercial) y así como también en el desarrollo de dispositivos o la gestión de recursos usando objetos inteligentes en un ambiente.\\
 
Por último la esta herramienta se espera que pueda servir como prueba de concepto y como marco de desarrollo para en principio ser usado dentro del Laboratorio de Internet de las Cosas de la Escuela de Computación de la Facultad de Ciencias de la Universidad Central de Venezuela, en aras de permitir una experiencia más profunda en el mundo de los dispositivos IoT. 

\section{Trabajos Futuros}
\lhead[\thepage]{\thesection. Trabajos Futuros}
Este trabajo fue pensado y diseñado para ser modular, pues el mundo del Internet de las cosas es bastante amplio y cambiante, por lo que para que este software pueda crecer organicamente y adaptarse a las necesidades de los usuarios, siendo incluso pilar fundamental durante cada etapa de este trabajo la utilización de tecnologías de software libre, que permitan extender características mas allá de las originales si así se desea.\\

Uno de los aspectos de mejora que tiene el desarrollo es la capacidad de poder utilizar y conectarse a dispositivos que utilicen otros protocolos y estándares distintos a MQTT de forma de abarcar una cantidad mayor de dispositivos y que este sea un hub para casi cualquier tipo de dispositivo IoT futuro.\\

Otra característica del software donde se puede mejorar es en la interfaz gráfica de la aplicación web HAMACA. De momento, la aplicación cuenta con una serie de elementos que permiten su correcta visualización en el computador pero se sugiere que para el futuro se tengan en cuenta interfaces mejor adaptadas para pantallas móviles así como también dentro de las herramientas integradas.\\

Por otro lado, si bien es cierto que todos los elementos desarrollados tienen una serie de funcionalidades de seguridad, se pueden agregar nuevos elementos y estándares que vayan surgiendo en el futuro para blindar la aplicación, teniendo en cuenta que se utilizan datos que pueden ser sensibles o de alto valor y causar daños en manos inescrupulosas si estos no están bien asegurados. Se sugiere examinar las posibilidades de extender este apartado tanto desde el punto de vista de las redes como de dispositivos también, el como mitigar vulnerabilidades inherentes al uso de un sistema web. 