%-----------------------------------------------------------------------------
%	 Marco Introductorio
%-----------------------------------------------------------------------------

\lhead[\thepage]{Marco Introductorio \thechapter. \rightmark}
\rhead[Marco Introductorio \thechapter. \leftmark]{\thepage}

%	Capitulo 1: Marco Introductorio
\chapter{Introducción}
\markboth{Introducción}{Introducción}
En general la adopción de nuevas tecnologías suele ocurrir de manera dispar. En algunas ocasiones la adopción es lenta y paulatina lo cual permite que pueda madurar en los diversos entornos en donde se implantan así como también permite crear formas ordenadas y planificadas de crecimiento de los elementos que se encuentran involucrados. Por otro lado también existen tecnologías que debido a su rápido crecimiento hacen que las personas y organizaciones deban adaptarse y ser flexibles en la manera en que se piensa se deben usar los avances, así como también la gama completa de oportunidades y debilidades que representa su uso.\\ 

Una de esas tecnologías que ha cambiado la manera en que los seres humanos actuamos es el computador personal. Con el acceso a una plataforma tan poderosa, la capacidad de poder automatizar elementos de la vida cotidiana y de procesos complejos en las industrias, se tiene la receta para ser una de las herramientas más importantes que haya creado el hombre. \\

Otra tecnología que ha cambiado al mundo es la capacidad de acceder y compartir información a través de una red. Su evolución a lo largo del tiempo a lo que ahora es el internet ha sido uno de los avances cruciales en la historia. No es una tecnología reciente, pero se ha masificado y democratizado su acceso de tal forma que es un aspecto omnipresente para mas de la mitad de la población mundial. Las diversas plataformas que se apoyan en la "red de redes" nos han ayudado a masificar la adopción de otro conjunto enorme de otras tecnologías, pues su flexibilidad y la madurez de los procesos que involucran la capacidad de conexión es catalizador de oportunidades para resolver problemas.\\

Si juntamos los aspectos de computo y conectividad vemos que de manera disruptiva actualmente se tiene la oportunidad de mejorar y automatizar muchos de los procesos que antes por costo, logística o complejidad eran difíciles de llevar a cabo. El crecimiento en la información y en masificación de artefactos y elementos que obtienen datos de su entorno proveen a los involucrados de una nueva visión del funcionamiento de las cosas que no era posible. Es esta revolución de la información la llamamos ``internet de las cosas" (o IoT por sus siglas en ingles) y es una tendencia en la tecnología en pleno crecimiento y que seguirá creciendo de manera activa en los años por venir.\\ 

Sin embargo como en casi todo nuevo avance en la tecnología, no ha venido sin presentar retos y dificultades propias. El gran volumen de información generado de manera automatizada, el control y monitoreo de dispositivos y artefactos a lo largo y ancho de complejos sistemas y y nuevos flujos automáticos donde antes no eran posibles de realizar hacen cada vez mas difícil el poder tener un panorama claro de las operaciones de estos sistemas por lo que se requieren de infraestructuras, plataformas y desarrollos nuevos para poder mejorar los aspectos de adopción mas ordenada de una forma tan nueva de hacer las cosas. Es así como nace la propuesta de comenzar a realizar la integración de tecnologías probadas que juntas puedan dar un mejor panorama en la observación y control de elementos de las operaciones y acciones que llevamos a cabo de manera automatizada en nuestro día a día.\\

En el siguiente trabajo de investigación se presenta una propuesta de un software que sea capaz de brindar la capacidad de mostrar datos en tiempo real e histórica provenientes de sensores y actuadores de dispositivos basados en el internet de las cosas, así como también la capacidad de crear flujos de automatización y control de los mismos, de forma que sea una plataforma centralizada para la gestión de los dispositivos IoT dentro de un ambiente en específico.\\

En el capitulo dos presentamos de una manera detallada el problema de investigación, los antecedentes y motivaciones que llevan a examinar este problema desde el punto de vista investigativo, así como cuales son los principios que justifican indagarlo. Con ese conocimiento podemos presentar una solución en donde se tiene el alcance del proyecto, así como también de plantean los objetivos generales y específicos con los que abordaran del punto de vista metodológico.\\

En los capítulos tres y cuatro se presentan los conceptos teóricos requeridos para abarcar los procesos mencionados previamente, en un primer momento tomando el concepto de internet de las cosas de manera clara y las ventajas y desventajas que ha tenido la adopción de este tipo de tecnologías en la sociedad y luego presentando un panorama sobre las herramientas de visualización y de control existentes y como el enfoque adecuado puede ayudar a cerrar la brecha entre los datos que se van generando y las gestión de dispositivos que son cada vez mas omnipresentes como complejos.\\

Para los capítulos cinco, seis y siete presentamos el marco metodológico utilizado para diseñar, crear, probar y validar el funcionamiento integral de los componentes desarrollados con el fin de poder cumplir con los objetivos planteados, incluyendo los posibles escenarios donde este proyecto puede dar un valor agregado a las estructuras existentes. \\
 
Por último, los capítulos ocho y nueve nos dan la presentación de los resultados finales tras el desarrollo, entendiendo las circunstancias con la que se desarrollo el problema y sugiriendo una serie de trabajos futuros pueden llevarse a cabo a partir de esta base. Ademas se establecen las conclusiones finales a las que se llegan marcando la tesis mencionada después de todo el trabajo investigativo realizado. \\
