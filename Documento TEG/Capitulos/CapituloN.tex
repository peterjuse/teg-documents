%-----------------------------------------------------------------------------
%	Capitulo N
%-----------------------------------------------------------------------------

%borrar esta sección si no hace nada 
%\lhead[\thepage]{CAPÍTULO \thechapter. \rightmark}
%\rhead[CAPÍTULO \thechapter. \leftmark]{\thepage}
%borrar esta sección si no hace nada 

%Nombre del Capitulo
\chapter{Nombre del capitulo}
\markboth{Nombre del capitulo}{Nombre del capitulo}

%Sección
\section{Nombre de la sección}
\lhead[\thepage]{\thesection. Nombre de la sección}

%En caso de una imagen colocarla de la siguiente manera
%\begin{figure}
%\centering
%\includegraphics[width=0.7\textwidth]{Figuras/nombre_de_la_figura.ext}
%\caption{\label{fig:etiqueta_de_la_figura}Titulo de la figura.}
%\end{figure}

%En caso de hacer referencia a una imagen hacerlo de la siguiente forma
%\textit{figura \ref{fig:etiqueta_de_la_figura}}

%En caso de hacer una cita y que esta se encuentre en la bibliografía
%\cite{etiqueta_de_la_referencia}


%-----------------------------------------------------------------------------
%	NOTA A LA HORA DE AGREGAR BIBLIOGRAFIAS REFERENCIAS.BIB
%-----------------------------------------------------------------------------

%ARTICLE: Un artículo de un periódico o una revista. Campos requeridos: author,title, journal, year. Campos opcionales: volume, number, pages, month, note.

%BOOK: Un libro con una editorial explícita. Campos requeridos: author o editor,title, publisher, year. Campos opcionales: volume o number, series, address,edition, month, note.

%BOOKLET: Un trabajo impreso y distribuido, pero que no tiene una editorial o institución responsable. Campos requeridos: title. Campos opcionales: author, howpublished, address, month, year, note.

%INBOOK: Una parte de un libro, como un capítulo, una sección, un rango de páginas,etc. Campos requeridos: author o editor, title, chapter o pages, publisher, year.Campos opcionales: volume o number, series, type, address, edition, month, note.

%INCOLLECTION: Una parte de un libro con título propio. Campos requeridos: author, title, booktitle, publisher, year. Campos opcionales: editor, volume o number,series, type, chapter, pages, address, edition, month, note.

%INPROCEEDINGS: Un artículo de las memorias de un congreso. Campos requeridos: author, title, booktitle, year. Campos opcionales: editor, volume o number, series, pages, address, month, organization, publisher, note. 

%MANUAL: Documentación técnica. Campos requeridos: title. Campos opcionales: author, organization, address, edition, month, year, note. 

%MASTERSTHESIS: Una tesis de maestría. Campos requeridos: author, title, school,year. Campos opcionales: type, address, month, note.

%MISC: Para cuando el resto falla. Campos requeridos: Ninguno. Campos opcionales:author, title, howpublished, month, year, note.

%PHDTHESIS: Tesis de doctorado. Campos requeridos: author, title, school, year. Campos opcionales: type, address, month, note.

%PROCEEDINGS: Las memorias de un congreso. Campos requeridos: title, year. Campos opcionales: editor, volume o number, series, address, month, organization, publisher, note.

%TECHREPORT: Un informe publicado por una institución. Campos requeridos: author,title, institution, year. Campos opcionales: type, number, address, month, note.

%UNPUBLISHED: Un documento (inédito), con un autor y un título, pero que no ha sido formalmente publicado. Campos requeridos: author, title, note. Campos opcionales: month, year.