%-----------------------------------------------------------------------------
%	Marco Metodológico: Fundational Methodology for Data Science
%-----------------------------------------------------------------------------

\lhead[\thepage]{Marco Metodológico \thechapter. \rightmark}
\rhead[Marco Metodológico \thechapter. \leftmark]{\thepage}

%	Capitulo 3: Marco Metodológico
\chapter{Marco Metodológico}
\markboth{Marco Metodológico}{Marco Metodológico}

%	Sección uno: Fundational Methodology for Data Science
\section{Fundational Methodology for Data Science}
\lhead[\thepage]{\thesection. Fundational Methodology for Data Science}

La ciencia de datos como campo que busca resolver problemas y dar respuesta a preguntas a través del análisis de datos es una practica que no es nueva y que debe proveer estrategias eficaces para poder obtener resultados no solamente precisos, sino también acordes con el método científico de una investigación y que permitan a las organizaciones utilizar lo obtenido para tomar acciones que mejoren los resultados futuros.\\

Con la cada vez mayor cantidad de tecnologías para analizar datos y construir modelos, existe la necesidad de tener un enfoque que ayude a la automatización de muchos de los pasos en su construcción, haciendo que tecnologías de aprendizaje automatizado sean mas accesibles a quienes carecen de habilidades estadísticas.\\

A medida que las capacidades de análisis de datos se vuelven más accesibles, los científicos necesitan una metodología fundacional capaz de proporcionar una estrategia de guía, independientemente de las tecnologías, volúmenes de datos o enfoques involucrados. Para ello IBM ideo una metodología para abordar los problemas de ciencia de datos en 10 pasos , que abarca tanto tecnologías como enfoques ```de arriba hacia abajo'' o  ``de abajo hacia arriba''.\\

Esta metodología tiene algunas similitudes con metodologías utilizadas para la minería de datos, pero hace hincapié en varias de las nuevas prácticas en la ciencia de datos, tales como el uso de grandes volúmenes de datos, la incorporación de análisis de texto en los modelos de predicción y la automatización de algunos procesos.\cite{ibmfmfds} La metodología consta de 10 etapas que forman un proceso iterativo para el uso de datos para descubrir ideas:

\begin{enumerate}
\item Comprensión del Negocio: Cada proyecto comienza con la comprensión del negocio. Los patrocinantes de los negocios que necesitan la solución analítica desempeñan el papel más crítico en esta etapa al definir el problema, los objetivos del proyecto y los requisitos de la solución desde una perspectiva empresarial. Esta primera etapa establece las bases para una resolución exitosa del problema. Para ayudar a garantizar el éxito del proyecto, los patrocinadores deben estar involucrados en todo el proyecto para proporcionar experiencia en el dominio, revisar los resultados intermedios y garantizar que el trabajo permanezca en el buen camino para generar la solución deseada.
\item Enfoque analítico: Una vez el problema del negocio sea claramente especificado, el científico de datos puede definir el enfoque analítico para el problema.Eta etapa implica expresar el problema en el contexto de las técnicas estadísticas y de aprendizaje automatizado, para que la organización pueda identificar las más adecuadas para obtener el resultado deseado.
\item Requerimientos en los datos: El enfoque analítico elegido determina los requisitos en los datos, es decir, luego de definidos los métodos analíticos a utilizar, establecer que se requiere de los datos, formatos y representaciones, guiados por el conocimiento del dominio.
\item Recolección de datos: En la etapa inicial de recopilación de datos, los científicos de datos identifican y recopilan los recursos de datos disponibles ,sin importar su tipo (estructurados, no estructurados y semi-estructurados) relevantes para el dominio del problema. Normalmente, se debe elegir si realizan inversiones adicionales para obtener elementos de datos menos accesibles. Si hay vacíos en la recopilación de datos, el científico de datos puede tener que revisar los requisitos de datos en consecuencia y recopilar nuevos datos y/o en mayores cantidades. Mediante la incorporación de más datos, los modelos predictivos pueden ser más capaces de representar eventos raros como la incidencia de la enfermedad o el fracaso del sistema.
\item Entender los datos: Después de la recopilación de datos original, el utilizar técnicas de estadística descriptiva y de visualización para comprender el contenido de los datos, evaluar la calidad de los mismos y descubrir conocimientos iniciales sobre los ellos. La recolección de datos adicionales puede ser necesaria para llenar los vacíos.
\item Preparación de los datos: Esta etapa abarca todas las actividades para construir los datos que se utilizarán en la siguiente etapa de modelado. Las actividades de preparación de datos incluyen la limpieza de datos (tratar valores perdidos o no válidos, eliminar duplicados, formatear adecuadamente), combinar datos de múltiples fuentes (archivos, tablas, plataformas) y transformar datos en variables más útiles.
En un proceso denominado ingeniería de características, los científicos de datos pueden crear variables explicativas adicionales, también denominadas predictivas o características, a través de una combinación de conocimientos de dominio y las variables estructuradas existentes. La preparación de los datos suele ser el paso más importante y largo de un proyecto de ciencias de datos. En muchos dominios, algunos pasos de preparación de datos son comunes en diferentes problemas. Automatizando ciertas etapas de preparación de datos de antemano pueden acelerar el proceso minimizando el tiempo de preparación ad-hoc. Con los actuales sistemas de alto rendimiento y la funcionalidad analítica en los que se almacenan los datos, los científicos de datos pueden preparar datos con más facilidad y rapidez utilizando conjuntos de datos muy grandes.
\item Modelado: A partir de la primera versión del conjunto de datos preparado, la etapa de modelado se centra en el desarrollo de modelos predictivos o descriptivos de acuerdo con el enfoque analítico previamente definido. Con modelos predictivos, los científicos utilizan un conjunto de formación (datos históricos en los que se conoce el resultado del interés) para construir el modelo. El proceso de modelado es típicamente interactivo ya que las organizaciones obtienen ideas intermedias, lo que lleva a refinamientos en la preparación de los datos y la especificación del modelo. Para una técnica dada, los científicos de datos pueden probar múltiples algoritmos con sus respectivos parámetros para encontrar el mejor modelo para las variables disponibles.
\item Evaluación: Durante el desarrollo del modelo y antes del despliegue, el científico de datos evalúa el modelo para entender su calidad y asegurarse de que aborda adecuadamente y completamente el problema del negocio. La evaluación del modelo implica el cálculo de diversas medidas de diagnóstico y otros resultados tales como tablas y gráficos, lo que permite al científico de datos interpretar la calidad del modelo y su eficacia para resolver el problema. Para un modelo predictivo, los científicos de datos utilizan un conjunto de pruebas, que es independiente del conjunto de entrenamiento, pero sigue la misma distribución de probabilidad y tiene un resultado conocido. El conjunto de pruebas se utiliza para evaluar el modelo para que pueda refinarse según sea necesario. A veces el modelo final se aplica también a un conjunto de validación para una evaluación final. Además, los científicos de datos pueden asignar pruebas de significación estadística al modelo como prueba adicional de su calidad. 
\item Despliegue: Una vez que se ha desarrollado un modelo satisfactorio y es aprobado por los patrocinadores de negocios, se despliega en el entorno de producción o un entorno de prueba comparable. Por lo general, se despliega de forma limitada hasta que se ha evaluado completamente su rendimiento. El despliegue puede ser tan simple como generar un informe con recomendaciones o involucrarse como en un flujo de trabajo complejo y un proceso de puntuación gestionado por una aplicación personalizada. El despliegue de un modelo en un proceso empresarial operacional suele implicar grupos, habilidades y tecnologías adicionales dentro de la empresa.
\item Feedback: Mediante la recopilación de los resultados del modelo implementado, la organización obtiene retroalimentación sobre el rendimiento del modelo y su impacto en el entorno en el que se implementó. El análisis de esta retroalimentación permite a los científicos de datos refinar el modelo para mejorar su precisión y utilidad. Pueden automatizar parte o la totalidad de los pasos de recopilación de información y evaluación de modelos, refinamiento y redistribución para acelerar el proceso de actualización de modelos para obtener mejores resultados.
\end{enumerate}

El flujo de la metodología ilustra la naturaleza iterativa del proceso de resolución de problemas. A medida que los científicos de los datos aprenden más sobre los datos y el modelado, frecuentemente regresan a una etapa previa para hacer ajustes. Los modelos no se crean una sola vez, se despliegan y se dejan en su sitio tal cual; En cambio, a través de retroalimentación, refinamiento y redistribución, los modelos se mejoran continuamente y se adaptan a las condiciones en evolución. De esta manera, tanto el modelo como el trabajo detrás de él pueden proporcionar un valor continuo a la organización mientras se necesite la solución.\cite{ibmfmfds}


%	Sección dos: CRISP - DM
\section{Cross Industry Standard Process for Data Mining (CRISP-DM)}
\lhead[\thepage]{\thesection. Cross Industry Standard Process for Data Mining (CRISP-DM)}

Cross Industry Standard Process for Data Mining abreviado como CRISP-DM, es una metodología para el proceso de la minería de datos que describe los enfoques comunes para orientar los proyectos, trabajos o investigaciones que se lleva a cabo una organización. Como metodología, incluye descripciones de las fases normales de un proyecto, las tareas necesarias en cada paso y una explicación de las relaciones entre las tareas. También se suele hablar de CRISP-DM como un modelo de proceso, pues ofrece un resumen del ciclo de vida de la minería de datos.\\

El ciclo de vida de CRISP-DM consta de seis fases que indican las dependencias mas importantes y frecuentes entre ellas\cite{ibmcrispdm} y son las siguientes:

\begin{enumerate}
\item Comprensión del negocio: Esta fase inicial se centra en la determinación y la comprensión de los objetivos y requisitos del proyecto desde una perspectiva empresarial, y luego convertir este conocimiento en una definición del problema de minería de datos, y un plan preliminar diseñado para alcanzar los objetivos.
\item Comprensión de datos: La fase de comprensión de datos de CRISP-DM implica estudiar más de cerca los datos disponibles para la minería. Este paso es esencial para evitar problemas inesperados durante la siguiente fase (preparación de datos) que suele ser la fase más larga de un proyecto. Esta fase comienza con una colección inicial de datos y con el objetivo de familiarizarse con los datos, los tipos de ellos, sus esquemas de codificación, describiendo métricas de calidad y precisión de los mismos, para descubrir las primeras señales dentro de los datos y detectar temas interesantes para poder formular hipótesis de información oculta.
\item Preparación de datos: La preparación de datos es uno de los aspectos más importantes y con frecuencia que más tiempo exigen en la minería de datos. De hecho, se estima que la preparación de datos suele llevar el 50\% al 70\% del tiempo y esfuerzo de un proyecto. Consiste en cubrir todas las actividades para construir el conjunto de datos para la minería.\cite{ibmcrispdm} Estas tareas son ejecutadas en múltiples oportunidades y sin un orden especifico. Las tareas incluyen selección y transformación de tablas, registros y atributos y limpieza de datos para las herramientas de modelado, derivación de nuevos atributos.
\item Modelado: En esta fase se seleccionan y aplican varias técnicas de modelado y se calibran los parámetros para obtener óptimos resultados. Hay varias técnicas que tienen requerimientos específicos para la forma de los datos, por lo que normalmente, los analistas de datos ejecutan varios modelos utilizando los parámetros por defecto y ajustan los parámetros o vuelven a la fase de preparación de datos para las manipulaciones necesarias por su modelo. Es extraño que las cuestiones relativas a la minería datos de una empresa se solucionen satisfactoriamente con un modelo y ejecución únicos.\cite{ibmcrispdm}
\item Evaluación: En esta etapa en el proyecto ha construido un modelo (o modelos) se evalúa  si los resultados obtenidos son acordes con el objetivo de la minería. En este punto, habrá completado la mayor parte de su proyecto de minería de datos.
\item Despliegue: Es el proceso que consiste en utilizar sus nuevos conocimientos para implementar las mejoras en su organización. Esta fase depende de los requerimientos, pudiendo ser simple como la generación de un reporte o compleja como la implementación de un proceso de explotación de información que atraviese a toda la organización.
\end{enumerate}

Aunque no es un paso formal dentro de CRISP-DM, es común en la mayoría de los casos se haga una revisión final del proyecto lo que le ofrece una oportunidad de formular sus impresiones finales e incorporar los conocimientos adquiridos durante el proceso de minería de datos. También es importante recordar que a pesar que estas etapas están claramente definidas  y ordenadas, es muy común que durante el proceso se retroceda a una etapa anterior con el fin de mejorar alguna característica deseada durante la investigación.

%	Sección tres: Scrum
\section{Scrum}
\lhead[\thepage]{\thesection. Scrum}

Scrum es una metodología ágil y flexible para gestionar el desarrollo de software, cuyo principal objetivo es maximizar el retorno de la inversión para su empresa (ROI). Se basa en construir primero la funcionalidad de mayor valor para el cliente y en los principios de inspección continua, adaptación, auto-gestión e innovación.\cite{scrumsofteng} En Scrum se aplican de manera regular un conjunto de buenas prácticas para trabajar colaborativamente, en equipo, y obtener el mejor resultado posible de un proyecto. Estas prácticas se apoyan unas a otras y su selección tiene origen en un estudio de la manera de trabajar de equipos altamente productivos.\cite{scrumproyecto}\\

Es parte de la filosofía de Scrum el poder realizar entregas parciales y regulares del producto final, priorizadas por el beneficio que aportan al receptor del proyecto. Por ello, Scrum está especialmente indicado para proyectos en entornos complejos, donde se necesita obtener resultados pronto, donde los requisitos son cambiantes o poco definidos, donde la innovación, la competitividad, la flexibilidad y la productividad son fundamentales.\\

En Scrum un proyecto se ejecuta en bloques temporales cortos y fijos (iteraciones que normalmente son de 2 semanas, aunque en algunos equipos son de 3 y hasta 4 semanas, límite máximo de feedback y reflexión). Cada iteración tiene que proporcionar un resultado completo, un incremento de producto final que sea susceptible de ser entregado con el mínimo esfuerzo al cliente cuando lo solicite. El proceso parte de la lista de objetivos/requisitos priorizada del producto, que actúa como plan del proyecto. En esta lista el cliente prioriza los objetivos balanceando el valor que le aportan respecto a su coste y quedan repartidos en iteraciones y entregas.\\  

La actividades que se llevan a cabo bajo la metodología de Scrum son las siguientes:

\begin{enumerate}
\item Planificación de la iteración: En el primer día de cada iteración, el equipo realza una reunión de planificación de la iteración. Esta etapa consiste en una primera fase de selección de requisitos, en la cual el cliente presenta al equipo la lista de requisitos priorizada del proyecto y se responden a las dudas que surgen sobre el dominio de las tareas a realizar y una segunda fase de planificación de la iteración, donde el equipo elabora la listas de tareas necesarias para desarrollar lo que se es requerido para la iteración, ademas de la estimación de esfuerzos por parte de los miembros del equipo.
\item Ejecución de la iteración: Cada día se realiza una reunión con todo el equipo de 15 minutos como máximo para inspeccionar el avance en la realización de las tareas, dependencias u actividades bloqueantes ye insumos pendientes para su culminación. Durante la iteración el Scrum Master (facilitador) se encarga de que el equipo pueda articularse y cumplir los compromisos adquiridos por el equipo.
\item Inspección y adaptación: El ultimo día de la iteración se realiza la reunión de revisión de la iteración. Esta se conforma de dos partes, la demostración en donde el equipo presenta al cliente el/los entregable(s) con los requisitos mínimos acordados y la retrospectiva en la que el equipo analiza si su manera de trabajar ha sido la adecuada, los problemas que han surgido, su solución, de forma que se pueda mejorar de manera continua la productividad.
\end{enumerate}